\documentclass[12pt,a4paper,spanish]{article} 
\usepackage{babel}
\usepackage [T1]{fontenc}
\usepackage [latin1]{inputenc}
\usepackage{graphicx} 
\usepackage{array}
	  \oddsidemargin 0in
      \textwidth 6.75in
      \topmargin 0in
      \textheight 10.0in
      \parindent 0em
      \parskip 2ex
\usepackage{anysize}
\marginsize{3cm}{2cm}{1.0cm}{1.0cm}

\pagestyle{plain}

\begin{document} 
\title{
\begin{table}[!h]
	\begin{tabular}{m{2cm}m{15cm}}
		\multicolumn{1}{l}{}
		 \includegraphics[scale=0.25, bb=0 0 0 0]{logo-fiuba.PNG} & 
		 \begin{center}
		 	\begin{LARGE}
				Universidad de Buenos Aires	\linebreak \linebreak		 							Facultad de Ingenier\'{i}a  \linebreak \linebreak
				7515 - Base de Datos \linebreak \linebreak
				1er. Cuatrimestre de 2010
			\end{LARGE}
		 \end{center}\\
\end{tabular}
\end{table}
\begin{Large}
 \begin{center}
		\underline{TP Base de Datos: SIGeek} \linebreak \linebreak
        Docente a cargo: Ing. Lucas Roman
\end{center}
\end{Large}
}
\date{}
\maketitle

\thispagestyle{empty}
\author{
\begin{Large}
\begin{center}
		\underline{Integrantes}  \linebreak 
\end{center}
\end{Large}
\begin{center}
	\begin{tabular}{|| c | c | c ||}
		\hline
		\begin{large}Apellido y Nombre\end{large} & 
		\begin{large}Padr\'{o}n Nro.\end{large} & 
		\begin{large}E-mail\end{large}\\
		\hline
		Bruno Tom�s & 88.449 & tbruno@gmail.com\\
		\hline
		Invernizzi Esteban Ignacio & 88.817 & invernizzie@gmail.com\\
		\hline
		Meller Gustavo Ariel  & 88.435 & gmeller@gmail.com\\
		\hline
		Rivero Hern\'an Javier & 88.455 & riverohernanj@gmail.com\\
		\hline
	\end{tabular}
\end{center}
}

\newpage
\setcounter{page}{1} 
\tableofcontents
\newpage

\section{Diagrama de Entidad - Interrelaci�n}

\section{Indicar dependencias de identidad y de existencia en el modelo}

\section{Supuestos que justifican el modelo (Hip�tesis) }


\section{Diccionario de datos}
	\subsection{Entidad 1}
		\subsubsection{Definici�n}
		\subsubsection{Especificaci�n de atributos}
		\subsubsection{Especificaci�n de identificador �nico}

	\subsection{Interrelaci�n 1}  
		\subsubsection{Definici�n}
		\subsubsection{Especificaci�n de atributos}
		\subsubsection{Especificaci�n de identificador �nico}
		
		
\section{Modelo Relacional}
	\subsection{Relaci�n 1}
		\subsubsection{Atributos}
		\subsubsection{Claves candidatas}
		\subsubsection{Clave primaria}
		\subsubsection{Claves for�neas}
		\subsubsection{Atributos que pueden tomar valores nulos}
		\subsubsection{Realice el diagrama del Modelo de Tablas}
		\subsubsection{Sentencias DDL}

Nota: en los casos en que existan diferentes alternativas para efectuar la transformaci�n de MER al modelo de tablas, elegir una �nica alternativa y enumerar las ventajas y desventajas de la alternativa elegida.


\end{document}