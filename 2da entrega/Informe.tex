\documentclass[12pt,a4paper,spanish]{article} 
\usepackage{babel}
\usepackage [T1]{fontenc}
\usepackage [latin1]{inputenc}
\usepackage{graphicx} 
\usepackage{verbatim}
\usepackage{array}
	  \oddsidemargin 0in
      \textwidth 6.75in
      \topmargin 0in
      \textheight 10.0in
      \parindent 0em
      \parskip 2ex
\usepackage{anysize}
\marginsize{3cm}{2cm}{1.0cm}{1.0cm}

\pagestyle{plain}

\begin{document} 
\title{
\begin{table}[!h]
	\begin{tabular}{m{2cm}m{15cm}}
		\multicolumn{1}{l}{}
		 \includegraphics[scale=0.5, bb=0 0 0 0]{logo-fiuba.PNG} & 
		 \begin{center}
		 	\begin{LARGE}
				Universidad de Buenos Aires	\linebreak \linebreak		 							Facultad de Ingenier\'{i}a  \linebreak \linebreak
				7515 - Base de Datos \linebreak \linebreak
				1er. Cuatrimestre de 2010
			\end{LARGE}
		 \end{center}\\
\end{tabular}
\end{table}
\begin{Large}
 \begin{center}
		\underline{TP Base de Datos: SIGeek} \linebreak \linebreak
        Docente a cargo: Ing. Lucas Roman
\end{center}
\end{Large}
}
\date{}
\maketitle

\thispagestyle{empty}
\author{
\begin{Large}
\begin{center}
		\underline{Integrantes}  \linebreak 
\end{center}
\end{Large}
\begin{center}
	\begin{tabular}{|| c | c | c ||}
		\hline
		\begin{large}Apellido y Nombre\end{large} & 
		\begin{large}Padr\'{o}n Nro.\end{large} & 
		\begin{large}E-mail\end{large}\\
		\hline
		Bruno Tom�s & 88.449 & tbruno88@gmail.com\\
		\hline
		Invernizzi Esteban Ignacio & 88.817 & invernizzie@gmail.com\\
		\hline
		Meller Gustavo Ariel  & 88.435 & gmeller@gmail.com\\
		\hline
		Rivero Hern\'an Javier & 88.455 & riverohernanj@gmail.com\\
		\hline
	\end{tabular}
\end{center}
}

\newpage
\setcounter{page}{1} 
\tableofcontents
\newpage

\section{Introducci�n}

	Esta segunda entrega consta de la resoluci�n y la expresi�n en lenguaje SQL de once consultas, contra un esquema relacional propuesto por la c�tedra. Las sentencias y cl�usulas a utilizar fueron restringidas a las presentadas por la c�tedra en la cartilla del apunte del lenguaje SQL.

\section{Consultas y resoluci�n}
	
	\begin{itemize}
	
		\item[Consulta 1] El promedio de mothers pedidas por pedido.\\
		\begin{verbatim}
Select sum(CANT_PEDIDA) / (count(distinct PEDIDO.NRO_PEDIDO) * count(distinct PEDIDO.NRO_PEDIDO))
From PEDIDO, ITEM_PEDIDO, SUBTIPO_COMPONENTE
Where ITEM_PEDIDO.SUBTIPO = SUBTIPO_COMPONENTE.SUBTIPO
  And TIPO = 'MOTHER'
		\end{verbatim}
		
		\item[Consulta 2] La raz�n social de los proveedores tales que existe un producto
que no comercializan.\\
		\begin{verbatim}
Select RAZON_SOCIAL
From PROVEEDOR pr1
Where exists (Select * From TIPO_COMPONENTE
              Where not exists (Select * From CATALOGO_PROVEEDOR, SUBTIPO_COMPONENTE
                                Where CATALOGO_PROVEEDOR.SUBTIPO = SUBTIPO_COMPONENTE.SUBTIPO
                                  And pr1.CUIT = CATALOGO_PROVEEDOR.CUIT
                                  And SUBTIPO_COMPONENTE.TIPO = TIPO_COMPONENTE.TIPO) )
		\end{verbatim}
		
		\item[Consulta 3] Un listado alfab�tico de subtipos de componentes a reponer indicando tipo de componente, subtipo de componente y razon_social de los proveedores que los comercializan.\\
		\begin{verbatim}
		\end{verbatim}
		
		\item[Consulta 4] La razon_social de los proveedores que comercializan todos los subtipos de componentes a reponer.
		\begin{verbatim}
		\end{verbatim}
		
		\item[Consulta 5] La razon_social, el nro_pedido y el subtipo de componente de los �tems de pedidos de subtipos de componente no comercializados por el proveedor del pedido.
		\begin{verbatim}
		\end{verbatim}
		
		\item[Consulta 6] La razon_social de los proveedores que comercializan exclusivamente un tipo de componente.
		\begin{verbatim}
		\end{verbatim}

		\item[Consulta 7] El subtipo de los componentes entregados exclusivamente por el proveedor 'Ceven S.A.'.
		\begin{verbatim}
		\end{verbatim}
		
		\item[Consulta 8] El subtipo de los componentes para los cuales el stock informado es distinto del real.
		\begin{verbatim}
		\end{verbatim}
		
		\item[Consulta 9] El nro_pedido y el subtipo de componentes de los �tems de pedidos cuya cantidad pedida supere a la cantidad pedida promedio de ese subtipo de componentes.
		\begin{verbatim}
		\end{verbatim}
		
		\item[Consulta 10] Para cada tipo de componente el subtipo de componente que tenga la mayor cantidad recibida.
		\begin{verbatim}
		\end{verbatim}
		
		\item[Consulta 11] El subtipo de los componentes entregados por todos los proveedores que comercializan todos los tipos de componentes.
		\begin{verbatim}
		\end{verbatim}
		
	\end{itemize}

     7. El subtipo de los componentes entregados exclusivamente por el proveedor 'Ceven S.A.'.

      8. El subtipo de los componentes para los cuales el stock informado es distinto del real.

      9. El nro_pedido y el subtipo de componentes de los �tems de pedidos cuya cantidad pedida supere a la cantidad

           pedida promedio de ese subtipo de componentes.

    10. Para cada tipo de componente el subtipo de componente que tenga la mayor cantidad recibida.

    11. El subtipo de los componentes entregados por todos los proveedores que comercializan todos los tipos de

          componentes.

\end{document}
